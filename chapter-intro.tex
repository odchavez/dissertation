\chapter{Introduction}

\section{Overview}

This thesis will focus on two areas: the modeling of text and user roles in networks, and the adaptation of monotonic learning methods to algorithmic fairness.

With the growth of social media, an increasing amount of new data generated occurs in a network context.  In Chapter \ref{ch:TopicBlockModel} I explore a model which combines the intuitive content description of topic modelling with the user archetyping of a stochastic blockmodel.  The combination allows us to describe contextual relationships of users (nodes) in the network and the content typical of the messages they exchange.

In Chapter \ref{ch:MonoFair} I explore the concept of monotonic individual fairness.  Individual fairness \citep{dwork2012fairness} formalizes the intuitive idea similar individuals should be treated similarly.  I extend this to the idea to formalize the idea that individuals' relative treatment should follow their relative qualification.  I accomplish this by identifying non-protected attributes which should have a monotonic relationship with the outcome and enforcing that the learned prediction function maintains this monotonicity.

I extend this concept in Chapter \ref{ch:SoftMonoFair} to the more realistic scenario in which monotonicity is derived from a sample of expert ratings.  This allows for more complex relationships between individuals; fairness might dictate that an increase in one attribute has a larger effect than a similar increase in another.  For example, a person a few violent felonies might considered more dangerous than a person with more non-violent misdemeanors.