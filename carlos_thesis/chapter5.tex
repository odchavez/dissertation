\chapter{Conclusions and future directions}
\label{ch:conclusions}

The common theme of the three major projects in this thesis was the
use of dependent priors for mixtures and random partitions. 
We discussed some motivating examples where the scientific research
questions naturally give rise to such dependent structures. 
Information on a specific form of dependence represents potentially
useful expert knowledge, which should be exploited when
available. It is still common practice, however, to ignore such domain
knowledge and proceed with default independent priors.
For the specific examples discussed in this thesis we constructed
suitable dependent models, developed practicable posterior inference
methods and demonstrated the proposed approaches in simulation studies
and in the actual applications.

Many open questions remain. For the application to cell lineage data,
the approach proposed in Chapter 3 can be characterized as an
empirical fit to the data with a model that respects the dependencies
that arise from the nature of the data. In future research we plan to
consider alternative generative models that mimick the actual biologic
process of how cells diversify. A generative model is used, for
example, in \cite{Shiffman18}, however still without using
restrictions and informative priors that would arise from the nature
of the data.

Also for the problem of matching patients and patient clusters with
representative cell lines that we considered in Chapter 4, many open
questions remain.
In the currently proposed model we match each patient cluster with one
representative cell line, implicitly allowing each cell line to be
paired with only one patient cluster. This restriction could be
removed, giving rise to a slightly different random structure.
Another aspect of the problem is that investigators have a preference
for reporting very distinct clusters of proteins (genes) and similarly
for the nested clusters of patients. That is, the desired summary of
the random partition should perhaps take into account preferences for
parsimony and interpretability. And such preferences could
alternatively already be included in the prior probability model.
One approach is the use of repulsive prior probability models that
favor very distinct clusters, for example the determinantal point
process (DPP) \citep{xu2016bayesian}. Similar issues arise with the applications in Chapters 2 and 3.





