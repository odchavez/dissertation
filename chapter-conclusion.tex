\chapter{Conclusion}

In this work, we have examined relational learning in two modes: performing inference on the individuals on the network, and designing systems that are fair for those individuals. 

In Chapter \ref{ch:TopicBlockModel}, we explored how to infer user communities based on the content and volume of their communications. This allows us to make inference about unseen relationships and messages, as well as to make inference about the larger organization represented in the network.

We conceptualized in Chapter \ref{ch:MonoFair} the resentment that individuals can experience when their relative treatment seems ``unfair'' relative to the treatment of other individuals, and explored how existing techniques for monotonic function learning can be used to avert such resentment.

Since such resentment can happen in more complex ways than monotonic functions can capture, we expanded the concept in Chapter \ref{ch:SoftMonoFair} to use a system of arbiter ratings to learn a preference function over individuals' attributes so that a system can prevent resentment relative to concepts of quality which are difficult to directly specify.